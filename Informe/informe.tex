%%%%%%%%%%%%%%%%%%%%%%%%%%%%%%%%%%%%%%%%%%%%%%%%%%%%%%%%%%%%%%%%%%%%%%%%%%%%%%%
% Definici�n del tipo de documento.                                           %
% Posibles tipos de papel: a4paper, letterpaper, legalpapper                  %
% Posibles tama�os de letra: 10pt, 11pt, 12pt                                 %
% Posibles clases de documentos: article, report, book, slides                %
%%%%%%%%%%%%%%%%%%%%%%%%%%%%%%%%%%%%%%%%%%%%%%%%%%%%%%%%%%%%%%%%%%%%%%%%%%%%%%%
\documentclass[a4paper,10pt]{article}


%%%%%%%%%%%%%%%%%%%%%%%%%%%%%%%%%%%%%%%%%%%%%%%%%%%%%%%%%%%%%%%%%%%%%%%%%%%%%%%
% Los paquetes permiten ampliar las capacidades de LaTeX.                     %
%%%%%%%%%%%%%%%%%%%%%%%%%%%%%%%%%%%%%%%%%%%%%%%%%%%%%%%%%%%%%%%%%%%%%%%%%%%%%%%

% Paquete para inclusi�n de gr�ficos.
\usepackage{graphicx}

% Paquete para definir la codificaci�n del conjunto de caracteres usado
% (latin1 es ISO 8859-1).
\usepackage[latin1]{inputenc}

% Paquete para definir el idioma usado.
\usepackage[spanish]{babel}


% T�tulo principal del documento.
\title{		\textbf{Trabajo Pr�ctico #0: Infraestructura b�sica}}

% Informaci�n sobre los autores.
\author{	Leandro Huemul Desuque, \textit{Padr�n Nro. 95836}                     \\
            \texttt{ desuqueleandro@gmail.com }                                              \\
            Cristian, \textit{Padr�n Nro. XXXXXX}                     \\
            \texttt{ mail@mail.com }                                              \\[2.5ex]
            \normalsize{1er. Cuatrimestre de 2017}                       \\
            \normalsize{66.20 Organizaci�n de Computadoras}                             \\
            \normalsize{Facultad de Ingenier�a, Universidad de Buenos Aires}            \\
       }
\date{}



\begin{document}

% Inserta el t�tulo.
\maketitle

% Quita el n�mero en la primer p�gina.
\thispagestyle{empty}

% Resumen
\begin{abstract}
Se desarroll� un programa en C que codifica y decodifica informaci�n en formato base 64.
El objetivo del presente trabajo fue familiarizarse con las herramientas b�sicas, tales como GXEmul y LaTeX.
\end{abstract}

\section{Introducci�n}
Se implement� en C un codificador/decodificador de informaci�n en base 64.
Base 64 es un sistema de numeraci�n posicional que usa 64 como base. Es la mayor potencia de dos que puede ser representada usando �nicamente los caracteres imprimibles de ASCII. Esto ha propiciado su uso para codificaci�n de correos electr�nicos, PGP y otras aplicaciones.
Como puede entenderse, se trata de un sistema simple por lo que resulta una buena elecci�n si el objetivo final no es construir un sistema elaborado sino familiarizarse con el software necesario para ello.

% \textit{66.20 Organizaci�n de Computadoras} ESTO PONE TEXTO EN CURSIVA

\section{Desarrollo}
\subsection{Documentaci�n del c�digo C}
La documentacion de las funciones se detalla por orden de aparici�n en el c�digo fuente.

\subsubsection{\texttt{help}}
\texttt{help} despliega la ayuda para el usuario final.

\subsubsection{\texttt{version}}
\texttt{version} informa la version del c�digo fuente.

\subsubsection{\texttt{TODO}}
\texttt{TODO} es una funci�n que TODO.

\subsubsection{\texttt{Especificaciones}}

%\\begin{figure}[!htp]
%\\begin{center}
%\\includegraphics[width=0.4\textwidth]{imagenStack.eps}
%\\end{center}
%\\caption{Stack de la funci�n vecinos} \label{fig:stack}
%\\end{figure}

FALTA, TODO.

\subsection{Dificultades}
FALTA, TODO.

\section{Compilaci�n}
FALTA, TODO.

Los argumentos utilizados para la compilaci�n son los siguientes:

\begin{description}
\item[-c] Compila el c�digo fuente pero no corre el linker. Genera el c�digo objeto.

\item[-o] Especifica el archivo de salida (ya sea un archivo objeto, ejecutable, ensamblado).

\item[-Wall] Activa los mensajes de warning.

\item[-I] Agrega el directorio especificado a la lista de directorios buscados para los archivos header
\end{description}

\section{Resultados}
FALTA, TODO.

\section{Conclusiones}
FALTA, TODO.

% Citas bibliogr�ficas.
\begin{thebibliography}{99}

\bibitem{GXemul} Sitio web de GXemul http://gxemul.sourceforge.net/

\end{thebibliography}

\end{document}
